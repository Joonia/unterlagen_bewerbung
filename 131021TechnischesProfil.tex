%%%%%%%%%%%%%%%%%%%%%%%%%%%%%%%%%%%%%%%%%
% "ModernCV" CV and Cover Letter
% LaTeX Template
% Version 1.1 (9/12/12)
%
% This template has been downloaded from:
% http://www.LaTeXTemplates.com
%
% Original author:
% Xavier Danaux (xdanaux@gmail.com)
%
% License:
% CC BY-NC-SA 3.0 (http://creativecommons.org/licenses/by-nc-sa/3.0/)
%
% Important note:
% This template requires the moderncv.cls and .sty files to be in the same 
% directory as this .tex file. These files provide the resume style and themes 
% used for structuring the document.
%
%%%%%%%%%%%%%%%%%%%%%%%%%%%%%%%%%%%%%%%%%

%----------------------------------------------------------------------------------------
%	PACKAGES AND OTHER DOCUMENT CONFIGURATIONS
%----------------------------------------------------------------------------------------

\documentclass[11pt,a4paper,sans]{moderncv} % Font sizes: 10, 11, or 12; paper sizes: a4paper, letterpaper, a5paper, legalpaper, executivepaper or landscape; font families: sans or roman

\moderncvstyle{classic} % CV theme - options include: 'casual' (default), 'classic', 'oldstyle' and 'banking'
\moderncvcolor{blue} % CV color - options include: 'blue' (default), 'orange', 'green', 'red', 'purple', 'grey' and 'black'
\usepackage{verbatim}
\usepackage{pdfpages}
\usepackage[absolute]{textpos}
\usepackage[utf8]{inputenc}

\usepackage[scale=0.75]{geometry} % Reduce document margins
\setlength{\hintscolumnwidth}{3.0cm} % Uncomment to change the width of the dates column
%\setlength{\makecvtitlenamewidth}{10cm} % For the 'classic' style, uncomment to adjust the width of the space allocated to your name

%----------------------------------------------------------------------------------------
%	NAME AND CONTACT INFORMATION SECTION
%----------------------------------------------------------------------------------------

\firstname{Jolanta} % Your first name
\familyname{Klocek} % Your last name

% All information in this block is optional, comment out any lines you don't need
\title{Technisches Profil}
%\address{Klagenfurter Strasse 45/7}{9300 St. Veit an der Glan}
%\mobile{0 664 6421402}
%\phone{(000) 111 1112}
%\fax{(000) 111 1113}
%\email{jolanta.klocek@gmx.de}
%\homepage{staff.org.edu/~jsmith}{staff.org.edu/$\sim$jsmith} % The first argument is the url for the clickable link, the second argument is the url displayed in the template - this allows special characters to be displayed such as the tilde in this example
%\extrainfo{additional information}
%\photo[70pt][0.4pt]{pictures/picture} % The first bracket is the picture height, the second is the thickness of the frame around the picture (0pt for no frame)
%\quote{"A witty and playful quotation" - John Smith}

%----------------------------------------------------------------------------------------

\begin{document}


%\newpage
%\thispagestyle{empty}	
%----------------------------------------------------------------------------------------
%----------------------------------------------------------------------------------------


\makecvtitle % Print the CV title

%----------------------------------------------------------------------------------------
%	EDUCATION SECTION Doktorstudium Elektronik  an  der  
%----------------------------------------------------------------------------------------

\section{KOMPETENZEN}


\cvitem{}
{
\begin{itemize}
	\item Instrumentelle Analytik (z.B. Röntgen-Photoelektronenspektroskopie, Photometrie, Atomspektroskopie, Potentiometrie, Chromatographie, UV, MS, NMR, IR, Raman-Spektroskopie)
	\item Chemische Untersuchungs – und Messverfahren 
	\item Versuchsplanung 
	\item Versuchsdurchführung  
	\item Datenakquise, -analyse und -auswertung
	\item Handhabung der komplexen Ultrahochvakuumsysteme (Röntgen-Photoelektronenspektroskopie, Rasterkraftmikroskopie und Synchrotron-basierte Techniken)
  \item Grenzflächenaktive Substanzen und disperse Systeme
	\item Sehr gute Kenntnisse in der Literaturrecherche
	\item Fachpublikationen erstellen 
	\item Sehr gute Erfahrung im Umgang mit chemischen Laborgeräten 
	\item Synthese (z.B. Atomlagenabscheidung, Sol-Gel-Verfahren)
	\item Chemische Isolierung (z.B. Extraktion)
	\item Kohlenstoff-Materialien
	\item Polymerchemie
	\item Anorganische Chemie 
	\item Englisch fließend 
\end{itemize}
}

\section{INTERRESE}


\cvitem{}
{
\begin{itemize}
	\item Chemie
	\item Spektroskopie
	\item Analytische Chemie  
	\item kohlenstoffhaltige Materialien 
	\item Forschung und Entwicklung  
	\item Kosmetik- und Waschmittelindustrie   
	\item Weiterbildung 
\end{itemize}
}

\section{AUSBILDUNG}

\cventry{10.2009 - 02.2012}{Promotionsstudium an der Brandenburgischen Technischen Universit\"at Cottbus}{}{}{}{Abschluss: Doktor der Naturwissenschaften (Dr. rer. nat.) \newline Schwerpunkte: Materialwissenschaften und Nanotechnologie}

\cventry{10.2003 - 07.2006}{Studium chemische Technologie an der Technischen Universit\"at Breslau}{}{}{}{Abschluss: Diplom des Ingenieurmagisters (Magister-Ingenieur) \newline Schwerpunkte: Grenzflächenaktive Substanzen und disperse Systeme }

\cventry{10.1999 - 08.2004}{Studium Chemie an der Universit\"at Breslau}{}{}{}{Abschluss: Diplom des Magisters \newline Schwerpunkte: allgemeine Chemie}


\section{SPRACHKENNTNISSE}
\cvitem{}
{
\begin{itemize}
	\item Deutsch verhandlungssicher
	\item Englisch verhandlungssicher
	\item Polnisch Muttersprache 
	\item Französisch mittel/fortgeschritten 
\end{itemize}
}



%----------------------------------------------------------------------------------------
%	WORK EXPERIENCE SECTION
%----------------------------------------------------------------------------------------

\section{PROJEKTE}

\cventry{03.2013-02.2014}{Wood Carinthian Competence Center, St.Veit/Glan }{}{}{}
{
\emph{Forschung und Entwicklung \newline Projektleiterin/ Labor} 
\newline \newline Entwicklung und Validierung neuer Analysemethoden
\newline
\begin{itemize}
	\item Entwicklung und Validierung neuer Analysemethoden im Bereich Reflektometrie und Tensiometer Messungen 
	\item Analyse der Messergebnisse
	\item Literaturrecherche
	\item Suche nach neuen Lösungen im Bereich Dispersionen
	\item Zusammenarbeit mit einem externen Industriepartner
	\item Teilnahme an nationalen und internationalen Konferenzen
	\item Statistische Datenanalyse
	\item Software Entwicklung
	\item Berichterstellung\newline    
\end{itemize} 
\emph{Technisches Umfeld \newline}
\begin{itemize}
	\item Matlab
	\item OriginPro
	\item Word
	\item Excel
	\item PowerPoint
	\item Outlook
\end{itemize}
}  


\cventry{2008-2012}{Brandenburgische Technische Universität, Cottbus}{}{}{}
{
\emph{Forschung und Entwicklung \newline Wissenschaftliche Mitarbeiterin, Doktorandin} 
\newline \newline Untersuchungen zur Herstellung und Stabilität von dünnen kohlenstoffhaltigen Filmen für Anwendung als Materialien mit niedriger Dielektrizitätskonstante
\newline
\begin{itemize}
	\item Röntgen-Photoelektronen-Spektroskopie (XPS)
	\item Röntgen-Nahkanten-Absorptions-Spektroskopie (NEXAFS)
	\item Fourier-Transform-Infrarot-Spektroskopie (FTIR)
	\item Analyse der Messergebnisse
	\begin{itemize}
		\item qualitative Analyse
		\item quantitative Analyse
	\end{itemize}
	\item Rasterkraftmikroskopie (AFM)
	\begin{itemize}
		\item Histogramm analyse
		\item Height-height correlation function analyse  
	\end{itemize}
	\item Kapazitäts-Spannungs-Messungen: CV  
	\item Atomlagenabscheidung (Atomic layer deposition) 
	\item Sol-Gel-Verfahren  
	\item Veröffentlichung der Arbeitsergebnisse in nationalen und internationalen Fachzeitschriften 
	\item Teilnahme an nationalen und internationalen Konferenzen\newline    
\end{itemize} 
\emph{Technisches Umfeld \newline}
\begin{itemize}
	\item OriginLab
	\item WinXAFS Software 
	\item Fityk: curve fitting and peak fitting software 
	\item HyperChem 
	\item ACD Labs software 
	\item Ultrahochvakuum Technik 
\end{itemize}
}  


\cventry{2007-2008}{Universität der Natur in Breslau, Polen}{}{}{}
{
\emph{Chemiker - Analytiker  \newline Technikerin } 
\newline 
\begin{itemize}
	\item selbständige  Führung des analytischen  Labors 
	\item Ausführung von chemischen  Analysen  mit klassischen Methoden und unter Anwendung der Methoden  der Instrumentalanalyse 
	\item Bearbeitung  von Messergebnissen  
	\item Suche nach  neuen  Lösungen  im Bereich  quantitative Analyse des Erdbodens und der Pflanzenmaterialien  
	\item Optimierung  der analytischen  Methoden
	\item Auswahl und Vervollständigung  des Laborgeräts 
	\item Versorgung und Ausrüstung des Labors 
	\item  Erteilung der Anweisungen  für die Personen, die im Labor chemische Analysen ausführen und Gewährung der Sicherheit am Arbeitsplatz 
\end{itemize} 
\emph{Technisches Umfeld \newline}
\begin{itemize}
	\item MS Office
	\item Flammen-Atomemissionsspektrometrie
	\item Photometrie 
	\item Kjeldahlsche Stickstoffbestimmung
\end{itemize}
}

\cventry{2005-2006}{Technische Universität Breslau, Polen}{}{}{}
{
\emph{Diplomarbeit \newline} 
\newline \newline Bewertung der rheologischen Eigenschaften der Dispersion von Aluminosilikaschichten
\newline
\begin{itemize}
	\item rheologische Untersuchungen von Aluminosilikate, die als Nanofüllstoffe in Polymer-Nanokomposite verwendet wurden 
	\item Untersuchung der Absorptionseigenschaften der organisch modifizierten Montmorillonite 
	\item Beschreibung der Ergebnisse in der Diplomarbeit \newline    
\end{itemize} 
\emph{Technisches Umfeld \newline}
\begin{itemize}
	\item MS Office 
\end{itemize}
}  

\cventry{2003-2004}{Universität Breslau, Polen}{}{}{}
{
\emph{Diplomarbeit \newline} 
\newline \newline Potentiometrische Titrationsuntersuchungen und UV-VIS-Spektroskopie an Nickel- und Diglycinekomplexen
\newline
\begin{itemize}
	\item Messungen der Strukturen von organometallischen Komplexverbindungen mit UV-Spektroskopie und potentiometrischer Titration  
	\item Datenauswertung und Interpretation 
	\item Beschreibung der Ergebnisse in der Diplomarbeit \newline    
\end{itemize} 
\emph{Technisches Umfeld \newline}
\begin{itemize}
	\item Super Quod 
	\item MS Office 
\end{itemize}
}  
\begin{comment}
\section{Publikationsliste}

\begin{itemize}
	\setlength{\itemsep}{1pt}		%it is setting
  \setlength{\parskip}{0pt}		%spaces betheen lines
  \setlength{\parsep}{0pt}		%to normal bechaviour as in default in itemize they are double spaced

\large 

\item T. Gotszalk, A. Marendziak, \textbf{K. Kolanek}, R. Szeloch, P. Grabiec, M. Zaborowski, P. Janus, and I. W. Rangelow, \textit{Scanning probe microscopy as a metrology method in micro-and nano-structure investigations}, Bulletin of The Polish Academy of Sciences - Technical Sciences, vol. 54, no. 1, pp. 19-23, 2006.

\item \textbf{K. Kolanek}, T. Gotszalk, M. Zielony, and P. Grabiec, \textit{Fabrication of micro- and nanostructures with local anodic oxidation by scanning probe microscopy}, Materials Science-Poland, vol. 26, no. 2, pp. 271-278, 2008.

\item K. Karavaev, \textbf{K. Kolanek}, M. Tallarida, D. Schmeisser, and E. Zschech, \textit{In-situ studies of ALD growth of hafnium oxide films}, Advanced Engineering Materials, vol. 11, no. 4, pp. 265-268, 2009.

\item \textbf{K. Kolanek}, P. Hermann, P. Dudek, T. Gotszalk, D. Chumakov, M. Weisheit, M. Hecker, E. Zschech, \textit{Local anodic oxidation by atomic force microscopy for nano-Raman strain measurements on silicon-germanium thin films}, Thin Solid Films, vol. 518, no. 12, pp. 3267-3272, 2010.

\item \textbf{K. Kolanek}, M. Tallarida, K. Karavaev, and D. Schmeisser, \textit{In situ studies of the atomic layer deposition of thin HfO$_{\textrm{2}}$ dielectrics by ultra high vacuum atomic force microscope}, Thin Solid Films, vol. 518, no. 16, pp. 4688-4691, 2010.

\item \textbf{K. Kolanek}, M. Tallarida, and D. Schmeisser, \textit{Atomic layer deposition of the HfO$_{\textrm{2}}$ investigated in situ by means of noncontact atomic force microscope}, Materials Science-Poland, vol. 28, no. 3, pp. 731-740, 2010.

\item \textbf{K. Kolanek}, M. Tallarida, M. Michling, K. Karavaev, and D. Schmeisser, \textit{Atomic layer deposition reactor for fabrication of metal oxides}, Physica Status Solidi (c), vol. 8, no. 4, pp. 1287-1292, 2011.


\item M. Tallarida, M. Weisheit, \textbf{K. Kolanek}, M. Michling, L. Starzyk, H-J. Engelmann, and D. Schmeisser, \textit{Atomic layer deposition of nanolaminate oxide films on Si}, Journal of Nanoparticle Research, vol. 13, no. 11, pp. 5975-5983, 2011. 


\item P. Hermann, M. Hecker, F. Renn, M. R\"olke, \textbf{K. Kolanek}, J. Rinderknecht, and L. M. Eng, \textit{Effects of patterning induced stress relaxation in strained SOI/SiGe layers and substrate}, Journal of Applied Physics, vol. 109, no. 12, p. 124513, 2011.


\item J. Klocek, \textbf{K. Kolanek}, and D. Schmeisser, \textit{Spectroscopic and atomic force microscopy investigations of hybrid materials composed of fullerenes and 3-aminopropyltrimethoxy-silane}, Journal of Physics and Chemistry of Solids, vol. 73, no. 5, pp. 699-706, 2012.

\item \textbf{K. Kolanek}, M. Tallarida, M. Michling, and D. Schmeisser, \textit{In situ study of the atomic layer deposition of HfO$_{\textrm{2}}$ on Si}, Journal of Vacuum Science \& Technology A: Vacuum, Surfaces, and Films, vol. 30, no. 1, p. 01A143, 2012.

\item J. Klocek, K. Henkel, \textbf{K. Kolanek}, E. Zschech, and D. Schmeisser, \textit{Spectroscopic and capacitance-voltage characterization of thin aminopropylmethoxysilane films doped with copper phthalocyanine, tris(dimethylvinylsilyloxy)-POSS and fullerene cages}, Applied Surface Science, vol. 258, no. 10, pp. 4213-4221, 2012.


\item J. Klocek, K. Henkel, \textbf{K. Kolanek}, K. Broczkowska, D. Schmeisser, M. Miller, and E. Zschech, \textit{Studies of the chemical and electrical properties of fullerene and 3-aminopropyltri-methoxysilane based low-k materials}, Thin Solid Films, vol. 520, no. 7, pp. 2498-2504, 2012.

\item J. Klocek, K. Henkel, \textbf{K. Kolanek}, E. Zschech, and D. Schmeisser, \textit{Annealing Influence on Siloxane-Based Materials Incorporated with Fullerenes, Phthalocyanines, and Silsesquioxanes} BioNanoScience, vol. 2, no. 1, pp. 52-58, 2012.


\item J. Klocek, \textbf{K. Kolanek}, K. Henkel, E. Zschech, and D. Schmeisser, \textit{Influence of the fullerene derivatives and cage polyhedral oligomeric silsesqiuoxanes on 3-aminopropyltrimethoxysilane based hybrid nanocomposites chemical, morphological and electrical properties}, Journal of Physics and Chemistry of Solids, vol. 74, no. 1, pp. 135-145, 2013.


\item \textbf{K. Kolanek}, M. Tallarida, and D. Schmeisser, \textit{Height distribution of atomic force microscopy images as a tool for atomic layer deposition characterization}, Journal of Vacuum Science \& Technology A: Vacuum, Surfaces, and Films, vol. 31, no. 1, p. 01A104, 2013.



\end{itemize}

\end{comment}



%\makeletterclosing % Print letter signature

%----------------------------------------------------------------------------------------

\end{document}
