%%%%%%%%%%%%%%%%%%%%%%%%%%%%%%%%%%%%%%%%%
% "ModernCV" CV and Cover Letter
% LaTeX Template
% Version 1.1 (9/12/12)
%
% This template has been downloaded from:
% http://www.LaTeXTemplates.com
%
% Original author:
% Xavier Danaux (xdanaux@gmail.com)
%
% License:
% CC BY-NC-SA 3.0 (http://creativecommons.org/licenses/by-nc-sa/3.0/)
%
% Important note:
% This template requires the moderncv.cls and .sty files to be in the same 
% directory as this .tex file. These files provide the resume style and themes 
% used for structuring the document.
%
%%%%%%%%%%%%%%%%%%%%%%%%%%%%%%%%%%%%%%%%%

%----------------------------------------------------------------------------------------
%	PACKAGES AND OTHER DOCUMENT CONFIGURATIONS
%----------------------------------------------------------------------------------------

\documentclass[11pt,a4paper,sans]{moderncv} % Font sizes: 10, 11, or 12; paper sizes: a4paper, letterpaper, a5paper, legalpaper, executivepaper or landscape; font families: sans or roman

\moderncvstyle{classic} % CV theme - options include: 'casual' (default), 'classic', 'oldstyle' and 'banking'
\moderncvcolor{blue} % CV color - options include: 'blue' (default), 'orange', 'green', 'red', 'purple', 'grey' and 'black'
\usepackage{verbatim}
\usepackage{pdfpages}
\usepackage{lipsum} % Used for inserting dummy 'Lorem ipsum' text into the template
\usepackage[absolute]{textpos}
\usepackage{lastpage}

\usepackage[scale=0.76]{geometry} % Reduce document margins
\setlength{\hintscolumnwidth}{3.2cm} % Uncomment to change the width of the dates column
%\setlength{\makecvtitlenamewidth}{10cm} % For the 'classic' style, uncomment to adjust the width of the space allocated to your name

%----------------------------------------------------------------------------------------
%	NAME AND CONTACT INFORMATION SECTION
%----------------------------------------------------------------------------------------

\firstname{Jolanta} % Your first name
\familyname{Klocek} % Your last name

% All information in this block is optional, comment out any lines you don't need
\title{LEBENSLAUF}
\address{Klagenfurter Str. 45/7}{9300 St. Veit an der Glan}
\mobile{+43 (0) 664 6421402} 
%\phone{(000) 111 1112}
%\fax{(000) 111 1113}
\email{jolanta.klocek@gmx.de}
%\homepage{staff.org.edu/~jsmith}{staff.org.edu/$\sim$jsmith} % The first argument is the url for the clickable link, the second argument is the url displayed in the template - this allows special characters to be displayed such as the tilde in this example
%\extrainfo{additional information}
\photo[70pt][0.4pt]{pictures/picture.jpg} % The first bracket is the picture height, the second is the thickness of the frame around the picture (0pt for no frame)
%\quote{"A witty and playful quotation" - John Smith}

%----------------------------------------------------------------------------------------

\begin{document}
\thispagestyle{empty}		%removes page numberin on the selected page

%----------------------------------------------------------------------------------------
%----------------------------------------------------------------------------------------
%\begin{comment}

\vspace*{30 mm} 				%*allows to use vspace at beginning of the page
\begin{center}
      \huge
			{
					
					Dr. rer. nat. Jolanta Klocek 
			}
\end{center}
\begin{center}
      \huge
			{
					Bewerbungsunterlagen 
			}
\end{center}
\begin{center}
      \huge
			{ 
					\includegraphics[width=5cm]{pictures/picture.jpg} 
			}
\end{center}
\vspace{20 mm}
\large 
Inhalt:
\begin{enumerate}
  \item Lebenslauf 
  \item Technisches Profil 
  \item Doktordiplom  
	\item Hochschuldiplom (Universit\"at Breslau) 
	\item Hochschuldiplom (Technische Universit\"at Breslau)
	\item Arbeitszeugnis Universit\"at der Natur in Breslau 
	\item Arbeitszeugnis Brandenburgische Technische Universit\"at  
	\item Arbeitszeugnis Kompetenzzentrum Holz GmbH  
\end{enumerate}
%\end{comment}

\newpage
%\thispagestyle{empty}	
%----------------------------------------------------------------------------------------
%----------------------------------------------------------------------------------------


\makecvtitle % Print the CV title

%----------------------------------------------------------------------------------------
%	EDUCATION SECTION Doktorstudium Elektronik  an  der  
%----------------------------------------------------------------------------------------

\section{Ausbildung}

%\cventry{10.2003 - 05.2008}{Doktorstudium Elektronik  an  der  Technischen  Universitat  Breslau (Wroclaw)}{Abteilung: Mikrosysteme, Elektronik und Photonik}{Spezialfach: Nanotechnologie}{Abschluss: Doktor der technischen Wissenschaften}  % Arguments not required can be left empty
%\cventry{\tiny 10.2003 - 05.2008}{Doktorstudium Elektronik  an  der  Technischen  Universitat  Breslau}{}{}{}}{} 
\cventry{10.2009 - 02.2012}{Promotionsstudium an der Brandenburgischen Technischen Universit\"at Cottbus}{}{}{}{Lehrstuhl: Angewandte Physik/Sensorik \newline Abschluss: Doktor der Naturwissenschaften (Dr. rer. nat.)}

\subsection{Dissertation}

\cvitem{Thema}{\emph{Untersuchungen zur Herstellung und Stabilit\"at von d\"unnen kohlenstoffhaltigen Filmen und ihre m\"ogliche Anwendung als Materialien mit niedriger Dielektrizit\"atskonstante}}

%%%%%%%%%%%%%%%%%%%%%%%%%%%%%%%%%%%%%%%%%%

\cventry{10.2003 - 07.2006}{Studium chemische Technologie an der Technischen Universit\"at Breslau}{}{}{}{Abteilung: Chemie \newline Spezialfach: Grenzfl\"achenaktive Substanzen und disperse Systeme \newline Abschluss: Diplom des Ingenieurmagisters (Magister-Ingenieur)}

\subsection{Diplomarbeit}

\cvitem{Thema}{\emph{Bewertung der rheologischen Eigenschaften der Dispersion von Aluminosilikaschichten}}


\cventry{10.1999 - 08.2004}{Studium Chemie an der Universit\"at Breslau}{}{}{}{Abteilung: Chemie \newline Spezialfach: allgemeine Chemie \newline Abschluss: Diplom des Magisters}

\subsection{Diplomarbeit}

\cvitem{Thema}{\emph{Potentiometrische Titrationsuntersuchungen und UV-VIS-Spektroskopie an Nickel- und Diglycinekomplexen}}




%%%%%%%%%%%%%%%%%%%%%%%%%%%%%%%%%%%%%%% Abitur

\cventry{09.1995 - 06.1999}{Stefan-Wyszynski-Allgemeinbildende Oberschule in Staszow}{}{}{} {Abschluss: Abitur}



%----------------------------------------------------------------------------------------
%	WORK EXPERIENCE SECTION
%----------------------------------------------------------------------------------------

\section{Berufserfahrung}

%\subsection{Vocational}

\cventry{03.2013 - 02.2014}{Projektleiterin/ Labor, Wood Carinthian Competence 
Center, St.Veit/Glan}{}{}{}{Entwicklung und Validierung neuer Analysemethoden, Durchf\"uhrung von Tests 
und Experimente im Labor, Matlab Software-Entwicklung} 

%------------------------------------------------

\cventry{10.2008 - 06.2009}{Wissenschaftliche Mitarbeiterin an der Brandenburgischen Technischen Universit\"at Cottbus}{}{}{}{\emph{Lehrstuhl Angewandte Physik/Sensorik} \newline Detaillierte Untersuchungen zur Herstellung und Stabilit\"at von d\"unnen kohlenstoffhaltigen Filmen mit niedriger Dielektrizit\"atskonstante}

%------------------------------------------------

\cventry{01.2007 - 08.2008}{Chemiker - Analytiker an der Universit\"at der Natur in Breslau, Polen}{}{}{}{Selbstst\"andige F\"uhrung des analytischen Labors}

\cventry{10.2005 - 06.2008}{Chemielehrerin in der Privatschule SIGMA in Breslau, Polen}{}{}{}{Vorbereitung von Sch\"ulern auf ihre Abschlusspr\"ufung und des Abiturs der Chemie}

\cventry{10.2004 - 07.2006}{Konsultant f\"r Kundenservice in der Firma IPT in Breslau, Polen}{}{}{}{Telefonische Kundenberatung} 

\cventry{07.2005 - 08.2005}{Praktikantin bei PZ Cussons AG in Breslau, Polen}{}{}{}{Qualitative Analyse des Waschmittels} 



%----------------------------------------------------------------------------------------
%	AWARDS SECTION
%----------------------------------------------------------------------------------------



%----------------------------------------------------------------------------------------
%	COMPUTER SKILLS SECTION
%----------------------------------------------------------------------------------------

\section{EDV-Kenntnisse}
\thispagestyle{empty}	

\cvitem{Software}{MS Office, LibreOffice, HyperChem, Microcal Origin, ACD Labs Software}
\cvitem{Programmier- sprachen}{MatLab}
\cvitem{Betriebsysteme}{Windows 2000/XP/Vista/7, Linux/Ubuntu}


%----------------------------------------------------------------------------------------
%	COMMUNICATION SKILLS SECTION
%----------------------------------------------------------------------------------------

\section{Fremdsprachenkenntnisse}

\cvitem{Polnisch}{Muttersprache}
\cvitem{Deutsch}{verhandlungssicher}
\cvitem{Englisch}{verhandlungssicher}
\cvitem{Franz\"osisch}{mittel fortgeschritten} 

%----------------------------------------------------------------------------------------
%	LANGUAGES SECTION
%----------------------------------------------------------------------------------------

\section{Pers\"onliche St\"arken}

%\renewcommand{\listitemsymbol}{-~} % Changes the symbol used for lists

\cvlistdoubleitem{Kreativit\"at}{Teamf\"ahigkeit}
\cvlistdoubleitem{Auffassungsf\"ahigkeit}{Flexibilit\"at}
\cvlistdoubleitem{Lernbereitschaft}{Analyse- und Probleml\"osef\"ahigkeit}
\cvlistdoubleitem{Selbst\"andiges Arbeiten}{Organisationsf\"ahigkeit}


%----------------------------------------------------------------------------------------
%	INTERESTS SECTION
%----------------------------------------------------------------------------------------

\section{Interessen und Hobbies}

%\renewcommand{\listitemsymbol}{-~} % Changes the symbol used for lists

\cvlistdoubleitem{Reisen}{Schwimmen} 
\cvlistdoubleitem{Kochen}{Fahrradfahren} 
\begin{comment}
\newpage
\section{Publikationsliste}

\begin{itemize}
  \setlength{\itemsep}{1pt}		%it is setting
  \setlength{\parskip}{7pt}	%spaces betheen lines
  \setlength{\parsep}{0pt}		%to normal bechaviour as in default in itemize they are double spaced

\large 

\item \textbf{J. Klocek}, K. Kolanek, and D. Schmeisser, \textit{Spectroscopic and atomic force microscopy investigations of hybrid materials composed of fullerenes and 3-aminopropyltrimethoxy-silane}, Journal of Physics and Chemistry of Solids, vol. 73, no. 5, pp. 699-706, 2012.

\item \textbf{J. Klocek}, K. Henkel, K. Kolanek, E. Zschech, and D. Schmeisser, \textit{Spectroscopic and capacitance-voltage characterization of thin aminopropylmethoxysilane films doped with copper phthalocyanine, tris(dimethylvinylsilyloxy)-POSS and fullerene cages}, Applied Surface Science, vol. 258, no. 10, pp. 4213-4221, 2012.

\item \textbf{J. Klocek}, K. Henkel, K. Kolanek, K. Broczkowska, D. Schmeisser, M. Miller, and E. Zschech, \textit{Studies of the chemical and electrical properties of fullerene and 3-aminopropyltri-methoxysilane based low-k materials}, Thin Solid Films, vol. 520, no. 7, pp. 2498-2504, 2012.

\item \textbf{J. Klocek}, K. Henkel, K. Kolanek, E. Zschech, and D. Schmeisser, \textit{Annealing Influence on Siloxane-Based Materials Incorporated with Fullerenes, Phthalocyanines, and Silsesquioxanes} BioNanoScience, vol. 2, no. 1, pp. 52-58, 2012.

\item \textbf{J. Klocek} \textit{Thin carbon containing films}, Dissertation, S\"udwestdeutscher Verlag f\"ur Hoch-schulschriften AG Co. KG, Saarbr\"ucken, ISBN 978-3-8381-3266-2, 2012.

\item \textbf{J. Klocek}, K. Kolanek, K. Henkel, E. Zschech, and D. Schmeisser, \textit{Influence of the fullerene derivatives and cage polyhedral oligomeric silsesqiuoxanes on 3-aminopropyltrimethoxysilane based hybrid nanocomposites chemical, morphological and electrical properties}, Journal of Physics and Chemistry of Solids, vol. 74, no. 1, pp. 135-145, 2013.

\end{itemize}
\end{comment}
% % % % % % % % % % % % % % % % % % % % % % % % % % % % % % % % % % % % % % % % %
% %ponizej zalaczane sa pliki pdf, ktore sa w katalogu podrzednym pliku zrodlowego .tex (/pdf/nazwa zalaczanego pliku.pdf) 
% % % % % % % % % % % % % % % % % % % % % % % % % % % % % % % % % % % % % % % % % 


\newpage
\includepdf[pages={-},pagecommand={\thispagestyle{fancy}}]{pdf/131021TechnischesProfil(1).pdf}
\includepdf[pages={1,3,4},pagecommand={\thispagestyle{fancy}}]{pdf/Diplome_Beglaubigte_Alle_Klocek.pdf}
\includepdf[pages={1},pagecommand={\thispagestyle{fancy}}]{pdf/Beglaubigte_wysrol.pdf}
\includepdf[pages={-},pagecommand={\thispagestyle{fancy}}]{pdf/beurteilung_jola.pdf}
\includepdf[pages={-},pagecommand={\thispagestyle{fancy}}]{pdf/Jola_Beur.pdf} 

\end{document}
